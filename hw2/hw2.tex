\documentclass[letterpaper, 12pt]{article}
\usepackage{fullpage,latexsym,amsthm,latexsym,amssymb}
\usepackage{amstext,amsfonts,amsmath,graphicx}
\usepackage{multicol}
\usepackage[usenames,dvipsnames]{color}
\usepackage{hyperref}
\usepackage{tikz}
\usetikzlibrary{automata, positioning}

\oddsidemargin = -0.5 in
\addtolength{\textwidth}{0.8in}
\addtolength{\textheight}{0.2in}

%% Theorem statements %%
% THEOREMS -------------------------------------------------------
\newtheorem{theorem}{Theorem}[section]
\theoremstyle{definition}
\newtheorem{definition}{Definition}
\numberwithin{equation}{section}
%% MISC DEFINITIONS %%%%%%%%%%%%%%%%%%%%%%%%%%%%%%%%%%%

\newcommand{\ignore}[1]{}

\DeclareMathOperator{\shuffle}{shuffle}

\newcommand{\N}{\mathbb N}
\newcommand{\abs}[1]{\left \vert #1 \right\vert}
\newcommand{\abi}[1]{\textcolor{Red}{#1}}
\newcommand{\qtrinfo}{CS181 Winter 2019}
\newcommand{\remove}[1]{}
\newcommand{\headnote}{
%\begin{multicols}{2}\small
\begin{itemize}
\item Please write your student ID \textbf{and the names of anyone you collaborated with} in the spaces provided and attach this sheet to the front of your solutions. \textbf{Please do not include your name anywhere since the homework will be blind graded.}
\item An extra credit of \textbf{5\%} will be granted to solutions written using \LaTeX. Here is one place where you can create \LaTeX documents for free: \url{https://www.sharelatex.com/}. The link also has tutorials to get you started. There are several other editors you can use.
\item If you are writing solutions by hand, please write your answers in a neat and readable hand-writing.
\item Always explain your answers. When a proof is requested, you should provide a rigorous proof.
\item If you don't know the answer, write ``I don't know'' along with a clear explanation of what you tried. For example: ``I couldn't figure this out. I think the following is a start, that is correct, but I couldn't figure out what to do next. [[Write down a start to the answer that you are sure makes sense.]] Also, I had the following vague idea, but I couldn't figure out how to make it work. [[Write down vague ideas.]]'' At least 20\% will be given for such an answer.

Note that if you write things that do not make any sense, no points will be given.
\item The homework is expected to take anywhere between 8 to 14 hours. You are
advised to start early.
\item Submit your homework online on the course webpage on CCLE. You can also hand it in at the end of any class before the deadline.
\end{itemize}
%\end{multicols}
}

\newcommand{\hwhead}[2]{
	\raggedleft{Student ID: \underline{\hspace{3in}} \\ \medskip
              Collaborators: \underline{\hspace{3in}} \\ \medskip}
  \bigskip
  \begin{center}
  	{\LARGE{\qtrinfo \ -- Problem Set #1}}\\[0.3cm]
  	{\Large{Due #2}}
  \end{center}
  \bigskip
  \raggedright
  \headnote
}

%%%%%%%%%%%%%%%%%%%%%%%%%%%%%%%%%%%%%%%%%%%%%%%%%%%%%
\addtolength{\topmargin}{-1cm}
\addtolength{\textheight}{2cm}






\begin{document}
\hwhead{2}{Tuesday, February 5, 11:59 pm}


\begin{center}
\fbox{%
   \parbox{0.8\linewidth}{
Note: \textit{All questions in the problem sets are challenging; you should not expect to know how to answer any question before trying to come up with innovative ideas and insights to tackle the question. If you want to do some practice problems before trying the questions on the problem set, we suggest trying problems 1.17 and 1.23 from the book. Do not turn in solutions to problems from the book.}
   }%
}
\end{center}


\newpage
\begin{enumerate}

\item{\bf (20 points)} Let $L_1$ and $L_2$ be languages and define
	\begin{align*}
		\shuffle(L_1, L_2) = \{x_1\, y_1\, x_2\, y_2\,\dots\, x_n\, y_n\mid x_1\, \dots\, x_n\in L_1, y_1\,\dots\, y_n\in L_2\}.
	\end{align*}
 Show that if the language $L_1$ is not regular and $L_2$ is any language then the languages $\shuffle(L_1, L_2)$ and $\shuffle(L_1, \overline{L_2})$ cannot both be regular.	 Recall for any language  $L$, $\overline{L}= \Sigma^{*}\setminus L$ denotes the complement language of $L$.\\
\emph{Hint: Recall closure properties of regular languages.}

 \item (\textbf{40 points}) In this problem we investigate the limits of the Pumping Lemma as it was stated in class and look for an alternative that remedies one of these shortcomings.
 	\begin{enumerate}
 		\item {\bf (10 points)} Let $L_1$ be the language
		$$L_1 = \{a^ib^{p} \mid \text{$i\geq 0$ and $p$ is a prime}\}.$$
		Prove that the language $L_2 = b^* \cup L_1 $ satisfies the conditions of the Pumping Lemma.
	I.e.\ show that there exists a $q\in \N$ such that for every word $w\in L_2$ with $\abs{w}\geq q$ we can write $w=xyz$ such that $\abs{xy}\leq q$, $\abs y>0$, and for every $i\geq 0$, $xy^iz \in L_2$.
		\item {\bf (20 points)} Prove the following generalization of the Pumping Lemma:\\ Let $L$ be a regular language. There exists a $q\in \N$ such that for every $w\in L$ and every partition of $w$ into $w=xyz$ with $\abs y\geq q$ there are strings $a, b, c$ such that $y=abc$, $\abs b>0$, and for all $i\geq 0$, $xab^icz\in L$.
		\item {\bf (10 points)} Prove that the language $L_2$ is not regular.
	\end{enumerate}








\item (\textbf{40 points}) For a language $L$ over alphabet $\Sigma$, we define
$$L_{\frac{1}{3}-\frac{1}{3}} = \{xz \in \Sigma^*\ |\ \exists y \in \Sigma^* \text{ with } |x|=|y|=|z| \text{ such that } xyz \in L\}.$$
Prove that if $L$ is regular, then $L_{\frac{1}{3}-\frac{1}{3}}$ need not be regular. \\
\emph{Hint: Consider the language $0^*21^*$ and recall closure properties of regular languages}
\end{enumerate}


\end{document}

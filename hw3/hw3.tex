\documentclass[letterpaper, 12pt]{article}
\usepackage{fullpage,latexsym,amsthm,latexsym,amssymb}
\usepackage{amstext,amsfonts,amsmath,graphicx}
\usepackage{multicol}
\usepackage[usenames,dvipsnames]{color}
\usepackage{hyperref}
\usepackage{tikz}
\usetikzlibrary{automata, positioning}

\oddsidemargin = -0.5 in
\addtolength{\textwidth}{0.8in}
\addtolength{\textheight}{0.2in}

%% Theorem statements %%
% THEOREMS -------------------------------------------------------
\newtheorem{theorem}{Theorem}[section]
\theoremstyle{definition}
\newtheorem{definition}{Definition}
\numberwithin{equation}{section}
%% MISC DEFINITIONS %%%%%%%%%%%%%%%%%%%%%%%%%%%%%%%%%%%

\newcommand{\ignore}[1]{}

\DeclareMathOperator{\shuffle}{shuffle}

\newcommand{\N}{\mathbb N}
\newcommand{\abs}[1]{\left \vert #1 \right\vert}
\newcommand{\abi}[1]{\textcolor{Red}{#1}}
\newcommand{\qtrinfo}{CS181 Winter 2019}
\newcommand{\remove}[1]{}
\newcommand{\headnote}{
%\begin{multicols}{2}\small
\begin{itemize}
\item Please write your student ID \textbf{and the names of anyone you collaborated with} in the spaces provided and attach this sheet to the front of your solutions. \textbf{Please do not include your name anywhere since the homework will be blind graded.}
\item An extra credit of \textbf{5\%} will be granted to solutions written using \LaTeX. Here is one place where you can create \LaTeX documents for free: \url{https://www.overleaf.com/}. The link also has tutorials to get you started. There are several other editors you can use.
\item If you are writing solutions by hand, please write your answers in a neat and readable hand-writing.
\item Always explain your answers. When a proof is requested, you should provide a rigorous proof.
\item If you don't know the answer, write ``I don't know'' along with a clear explanation of what you tried. For example: ``I couldn't figure this out. I think the following is a start, that is correct, but I couldn't figure out what to do next. [[Write down a start to the answer that you are sure makes sense.]] Also, I had the following vague idea, but I couldn't figure out how to make it work. [[Write down vague ideas.]]'' At least 20\% will be given for such an answer.

Note that if you write things that do not make any sense, no points will be given.
\item The homework is expected to take anywhere between 8 to 14 hours. You are
advised to start early.
\item Submit your homework online on Gradescope. 
\end{itemize}
%\end{multicols}
}

\newcommand{\hwhead}[2]{
	\raggedleft{Student ID: \underline{\hspace{3in}} \\ \medskip
              Collaborators: \underline{\hspace{3in}} \\ \medskip}
  \bigskip
  \begin{center}
  	{\LARGE{\qtrinfo \ -- Problem Set #1}}\\[0.3cm]
  	{\Large{Due #2}}
  \end{center}
  \bigskip
  \raggedright
  \headnote
}

%%%%%%%%%%%%%%%%%%%%%%%%%%%%%%%%%%%%%%%%%%%%%%%%%%%%%
\addtolength{\topmargin}{-1cm}
\addtolength{\textheight}{2cm}






\begin{document}
\hwhead{3}{Monday, February 11, 11:59 pm}


\begin{center}
\fbox{%
   \parbox{0.8\linewidth}{
Note: \textit{Suggested practice problems from the book: 2.4 and 2.5. Please, do not turn in solutions to problems from the book.}
   }
}
\end{center}


\newpage
\begin{enumerate}
\item \textbf{(20 points).} Consider a binary operation $\nabla$ defined as follows: if $A$ and $B$ are two languages,
then $A \nabla B = \{xy \mid x \in A\text{, }y \in B\text{, and } \abs x = \abs y \}$. Prove that if $A$
and $B$ are regular languages, then $A \nabla B$ is a context-free language.


\item {\bf  (45 points).} This problem explores two related languages. Remember to use the ideas from part (a) in part (b).
\begin{enumerate}
	\item \label{first}\textbf{(20 points).} Show that the language
	\begin{align*}
		L_1 = \left\{ x\$ y \mid \text{$x, y\in \{0, 1\}^*$ and $x\neq y$} \right\}
	\end{align*}
over the alphabet $\Sigma=\{\$, 0, 1\}$ is a context-free language.
	\item \label{second}\textbf{(25 points).} Show that the language
\begin{align*}
	L_2 = \left\{ xy\mid \text{$x, y\in \{0, 1\}^*$, $\abs x = \abs y$, and $x\neq y$}  \right\}
\end{align*}
is a context-free language.
\end{enumerate}
\emph{Hint: Have non-determinism on your mind.}
\end{enumerate}


\end{document}

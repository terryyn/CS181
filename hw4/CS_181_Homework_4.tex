\documentclass[12pt]{article}
\usepackage{fullpage,latexsym,amsthm,latexsym,amssymb}
\usepackage{amstext,amsfonts,amsmath,graphicx}
\usepackage{multicol}
\usepackage[usenames,dvipsnames]{color}
\usepackage{hyperref}

\oddsidemargin = -0.5 in
\addtolength{\textwidth}{0.8in}
\addtolength{\textheight}{0.2in}

%% Theorem statements %%
% THEOREMS -------------------------------------------------------
\newtheorem{theorem}{Theorem}[section]
\theoremstyle{definition}
\newtheorem{definition}{Definition}
\numberwithin{equation}{section}
%% MISC DEFINITIONS %%%%%%%%%%%%%%%%%%%%%%%%%%%%%%%%%%%

\newcommand{\ignore}[1]{}

\newcommand{\abi}[1]{\textcolor{Red}{#1}}
\newcommand{\qtrinfo}{CS181 Winter 2019}
\newcommand{\remove}[1]{}
\newcommand{\headnote}{
%\begin{multicols}{2}\small
\begin{itemize}
\item Please write your student ID \textbf{and the names of anyone you collaborated with} in the spaces provided and attach this sheet to the front of your solutions. \textbf{Please do not include your name anywhere since the homework will be blind graded.}
\item An extra credit of \textbf{5\%} will be granted to solutions written using \LaTeX. Here is one place where you can create \LaTeX documents for free: \url{https://www.overleaf.com/}. The link also has tutorials to get you started. There are several other editors you can use.
\item If you are writing solutions by hand, please write your answers in a neat and readable hand-writing.
\item Always explain your answers. When a proof is requested, you should provide a rigorous proof.
\item If you don't know the answer, write ``I don't know'' along with a clear explanation of what you tried. For example: ``I couldn't figure this out. I think the following is a start, that is correct, but I couldn't figure out what to do next. [[Write down a start to the answer that you are sure makes sense.]] Also, I had the following vague idea, but I couldn't figure out how to make it work. [[Write down vague ideas.]]'' At least 20\% will be given for such an answer.

Note that if you write things that do not make any sense, no points will be given.
\item The homework is expected to take anywhere between 8 to 14 hours. You are
advised to start early.
\item Submit your homework online on Gradescope. 
\end{itemize}
%\end{multicols}
}

\newcommand{\hwhead}[2]{
	\raggedleft{Student ID: \underline{\hspace{3in}} \\ \medskip
              Collaborators: \underline{\hspace{3in}} \\ \medskip}
  \bigskip
  \begin{center}
  	{\LARGE{\qtrinfo \ -- Problem Set #1}}\\[0.3cm]
  	{\Large{Due #2}}
  \end{center}
  \bigskip
  \raggedright
  \headnote
}

%%%%%%%%%%%%%%%%%%%%%%%%%%%%%%%%%%%%%%%%%%%%%%%%%%%%%
\addtolength{\topmargin}{-1cm}
\addtolength{\textheight}{2cm}

\newcommand{\sbset}{\ensuremath{SUBSET_{TM}}}

\newcommand{\PDA}{\mathsf{PDA}}

\begin{document}
\hwhead{4}{Tuesday, February 26, 11:59 PM}

\newpage
\begin{enumerate}

	
		\item \textbf{(20 points)} Show that the set of three-dimensional coordinates $\{(x,y,z) | x, y, z \in \mathbb{Z}\}$ has size equal to $\mathbb{N}$.

	\item \textbf{(30 points)} Let us define Fast-Rewind Turing Machines (FRTM). They are similar to Turing Machines, except that the head of an FRTM is not allowed to move left one cell at a time.
	Instead, the head of the FRTM \emph{must} move left only all the way to the left-hand end of the tape (i.e. the first cell). The head of the FRTM can move right one step at a time, just like the usual Turing Machines.
	The transition function for the FRTM is of the form $\delta: Q \times \Gamma \rightarrow Q \times \Gamma \times \{R,FIRST\}$.

	Explain how to convert any Turing Machine into an FRTM.
	A full proof is not needed, you only need to give an explanation of the intuition behind why your transformation works.
	


\item \textbf{(50 points).} In class, we showed the existence of two kinds of infinities.
  Let $\Sigma = \{0,1\}$ and $\mathcal{L} = \{L \mid L \subseteq \Sigma^*\}$. We showed that
  $|\Sigma^*|$ is not the same as $|\mathcal{L}|$.

  We briefly describe the proof below and establish some notation. The proof
  proceeds by contradiction. Let $(\epsilon, 0, 1, 00, 01, 10, 11, \ldots)$ be
  a given enumeration of strings in $\Sigma^*$. We assume for the sake of
  contradiction that there exists some enumeration $(L_1, L_2, L_3, \ldots)$
  of languages in $\mathcal{L}$. Then we proceed by constructing a language
  $L^{DIAG} \in \mathcal{L}$ such that $\forall i, L^{DIAG} \not = L_i$ via
  Cantor's Diagonalization to establish a contradiction.
  \begin{enumerate}
    \item \textbf{(10 points)} Call $L_1^{DIAG} = L^{DIAG}$. Construct another language
    $L_2^{DIAG} \not = L_1^{DIAG}$ via diagonalization which is also not
    present in the enumeration $(L_1, L_2, L_3, \ldots)$. Thus $L_2^{DIAG}$
    would have also proved to us that $|\Sigma^*| \not = |\mathcal{L}|$.
    \item \textbf{(15 points)} Construct an infinite set of languages $\mathcal{L}^{DIAG}
     = \{L_1^{DIAG}, L_2^{DIAG}, \ldots, L_i^{DIAG}, \ldots \}$ via
     diagonalization by providing a description of $L_i^{DIAG}$ such that
     \begin{itemize}
       \item $ \forall j, j \not = i, L_i^{DIAG} \not = L_j^{DIAG} $.
       \item $ \forall j, L_i^{DIAG} \not = L_j$.
       \end{itemize}
       Formally prove your construction by induction.
     \item \textbf{(15 points)} Construct one more language
     $L^{SUPERDIAG}$ such that
     \begin{itemize}
       \item $ \forall j, L^{SUPERDIAG} \not = L^{DIAG}_j$ and
     \item $ \forall j, L^{SUPERDIAG} \not = L_j$
     \end{itemize}
     Briefly explain why your language satisfies the above properties.
     \item \textbf{(10 points)} Construct yet another language $L^{SUPERDIAG}_2$ that is different
     from $L^{SUPERDIAG}$ satisfying the same properties as part (c). Briefly
     explain your answer.
%     Construct another infinite set $\mathcal{L}'_{DIAG}$ that
%     satisfies conditions of part (b) and $\mathcal{L}'_{DIAG} \cup
%     \mathcal{L}_{DIAG}$ is empty. Formally prove your construction by induction.
%     \item Ponder about, you don't need to answer, the existence of a set
%     $\mathcal{L''}_{DIAG}$ such that for all $L \in \mathcal{L''}_{DIAG}$, for
%     all $j, L \not = L_j$, and $|\mathcal{L''}_{DIAG}| = \infty_2$.
  \end{enumerate}

\end{enumerate}



\end{document}